\documentclass{article}
\usepackage[utf8]{inputenc}
\usepackage{hyperref}
\title{diKTat}

\begin{document}

\maketitle

\newpage

\tableofcontents

\newpage

\section{Introduction}
\par
DiKTat - is a formal strict code style (\url{https://github.com/cqfn/diKTat}) and a linter with a set of rules that implement this code style. Basically, it is a collection of Kotlin code style rules implemented as AST visitors on top of KTlint framework (\url{https://github.com/pinterest/ktlint}). Diktat warns and fixes code style errors and code smells based on configuration file. DiKTat is a highly configurable framework, that can be extended further by adding custom rules. It can be run as command line application or with maven or gradle plugins. In this paper, we will explain how DiKTat works, describes advantages and disadvantages and how it differs from other static analyzers.

\newpage

\section{How does diKTat work}
\subsection{The idea}
\subsection{Rules description}
\subsection{Examples}

\newpage

\section{Analysis of similar projects}
\subsection{Ktlint}
\subsection{Detekt}
\newpage

\section{How to use diKTat}
\subsection{Maven}
\subsection{Gradle}
\subsection{CLI-application}
\subsection{How to configure ruleset}
\newpage

\section{DiKTat on big projects}

\newpage
\section{Appendix}
\subsection{Main components}
\subsubsection{Data diagram}
\subsubsection{Class diagram}
\subsubsection{Component view}

\subsection{Diagrams of system usage}
\subsubsection{Use-case diagram}
\subsubsection{Activity diagram}
\subsubsection{Data Flow Diagram}

\newpage
\section{References}
\end{document}
