Before continue, it is necessary to define some terms so that the reader correctly understands the context of what was written. 
The first and basic concept that should be introduced is \textbf{"rule"}. In diKTat, a \textbf{“rule”} is the logic described in a class, which checks a certain paragraph of code style for compliance with the code.
 You should also know that \textbf{set} -  is a well-defined collection of distinct objects, considered as an object in its own right. 
 \textbf{Ruleset}, in turn, is a set of such "rules". Suppose $W_i$ is set of checks of i-th rule and $F_i$ is set of fixers of i-th rule. Then $I_i = W_i \cup F_i $ is the set of checks and fixers of the i-th rule. Let R be a ruleset, therefore $R = \sum_{k=1}^n I_k$, where n is number of turned-on rules, $I_k$ is k-th rule.
 \textbf{Abstract syntax tree(AST)} is a tree representation of the abstract syntactic structure of source code written in a programming language. Each node of the tree denotes a construct occurring in the source code.
 \textbf{CICD} - continuous integration (CI) and continuous delivery (CD) is a methodology that allows application development teams to make changes to code more frequently and reliably.
 \textbf{KDoc} - is the language used to document Kotlin code (the equivalent of Java's JavaDoc).